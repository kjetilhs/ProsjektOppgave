%===================================== CHAP 5 =================================

\chapter{Conclusion and recommendation for further work}
This chapter will discus the result from the field tests, which will be used to conclude the performance of the individual \gls{rtk-gps} system. The last part of this chapter will include suggestion for further work regarding the design of a automatic net landing system.
\section{RTK-GPS lab testing}
It was seen in the lab testing that both the Piksi and \gls{rtklib} was able to get a fixed solution, although the \gls{rtklib} was faster. In the first test were the \gls{uav} was carried both system gave similar result during the first session \gls{uav}, however during the second the Piksi lost its fixed solution. \gls{rtklib} had also problem during this session, but unlike the Piksi managed to find a fixed solution again. The ability for the navigation system to quickly recover its integer solution is critical for a automatic net landing system.

The \gls{rtk-gps} system achieves centimeter level accuracy when stationary. When the \gls{uav} is in motion this will decrease down to decimeter accuracy. This can be assumed is the behaviour for the navigation system based on the stationary accuracy test, and the high precision in the position estimate. That should be good enough in a navigation system for a automatic landing system.

The output solution from \gls{rtklib} has a \acrfull{tow} value that is constant half a second delayed compared to the Piksi. It could be that \gls{rtklib} interpolate the microseconds of the \gls{tow} value from the different satellites, such that a control system do not get a delayed position estimate. However for a integrated navigation system this pose a problem. The integration becomes more difficult when it's unclear how old the position estimate is.

The tracking performance from the two receiver type indicate that the Ublox is superior to the Piksi. It always manage to track more satellites, and managed to track them longer.
\section{RTK-GPS in-flight test}
The flight test showed that keeping the fixed integer solution during a flight is difficult, most likely because of the dynamic behaviour of the \gls{uav}. The landing phase of the \gls{uav} operation should be kept independent from the rest of the operation. Therefore landing specific constraint cannot be imposed on the \gls{uav} behaviour. In order to then get a best relative position estimate the \gls{rtk-gps} must be able to recover it's fixed integer solution as quickly as possible while the \gls{uav} is in flight.

Before the take-off the Piksi had yet solved its integer ambiguity, while \gls{rtklib} had. That might be because of it kept track of fever satellites. However the navigation system must be able to resolve its integer ambiguity quickly to increase the probability of performing a safe and successfully automatic landing.

During the flight the Ublox lost track of more satellite then during the \gls{gps} lab tests. The antenna follows the orientation of the \gls{uav} and therefore some satellite will be blocked by the fuselage. This must be consider when designing a guidance system for automatic landing. To ensure that the antenna has a clear view of the sky, the \gls{uav} should try to have a maximum roll when the automatic landing system is engaged.

The velocity estimate from Piksi and \gls{rtklib} in the first test have similar estimate for the North and East component, but differ in the Down component. In the flight test the \gls{rtklib} altitude velocity was better, such that it can be used in a control system.
%
%
%
%
%
%
%
%DO NOT INCLUDE WHAT IS BELLOW BEFORE FURTHER WORK
%
%
%It might be that the float position estimate is good enough to perform a automatic landing. Further test are required.
%
%Rtklib managed to keep float/fixed solution during the landing.
%
%The rtklib was the only system able to get a fixed solution. Discus the performance of rtklib, whit the ublox receiver. 
%
%What can be concluded:
%
%Rtklib is able to faster find a fixed solution. Can be because of better receiver, better integer ambiguity resolution strategy, poorer validation of ambiguity, more efficient algorithm. Based on how many satellites the ublox receiver manage to track I will say that the receiver had a large part in shortening the time.
%
%When fixed there is not a large deviation between the two solutions. One difference that might cause problem in a guidance system is that it appear that the solution from rtklib is delayed. The \gls{tow} that's given with the solution do not include the millisecond value. 
%
%The Piksi is faster to calculate the position, but slower to resolve the integer ambiguity. 
%
%The conclusion of the different system. Piksi has better and more efficient solution software, however rtklib can use a superior \gls{gnss} receiver.
%
%Both system has a good estimation of the North and East velocity, however it appear that rtklib has problem estimating the vertical velocity.


\section{Further work}
The Ublox receiver and the \gls{gnss} antennas are able to receive both \gls{gps} and \gls{glonass} \gls{l1} signals, which should be exploited to increase the number of valid satellite available for the \gls{rtk-gps} system. This could also give better constellation geometry, and help the system resolve its integer ambiguity faster.

The \gls{rtk-gps} system must be integrated with the existing control and guidance system, and a test landing in a stationary net must be performed. 

A \gls{rtk-gps} system can use both the Piksi and \gls{rtklib} such that the automatic landing system has redundancy in position, and velocity estimation. A validation task must then be design to choose which of the system should send the rtkfix \gls{imc} message to the rest of the system.

A \gls{rtk-gps}/gls{ins} integration navigation system can be implemented in the automatic landing system. A \gls{rtk-gps}/gls{ins} integration was proposed in \citep{Spockeli}, which can be used.

A position estimation system that can accurately predict were the \gls{uav} will be a few seconds ahead of time, such that the \gls{uav} can better know were it is instead of were if was.

Setting constraints on the path such that the antenna has a clear view of the satellites at all time during the landing phase. 
\cleardoublepage