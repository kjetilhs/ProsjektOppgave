%===================================== CHAP 5 =================================

\chapter{Closing discussion and conclusion}
This chapter present the discussion on the result from the experimental tests, which will be used to conclude the performance of the individual \gls{rtk-gps} system. Then recommendation for further work is presented regarding the design of a automatic net landing system.
\section{Closing discussion and conclusion}
In the fist test both the \gls{rtklib} and Piksi managed to get a fixed solution, which gave similar result from both systems. However during the second run the Piksi lost its fixed integer ambiguity solution, while \gls{rtklib} managed to keep its fixed integer ambiguity solution. The ability for the navigation system to quickly recover its integer ambiguity solution is critical for the performance of a automatic net landing system.

The \gls{rtk-gps} system achieves centimetre level accuracy when stationary. Together with the high precision solution from the navigation system the accuracy of the relative position would be at a decimetre level, given that the integer ambiguity solution is fixed. With a float solution the accuracy will decrease, and it has yet to be determined if the degradation will make the \gls{uav} unable to perform a automatic landing in a net.

The output solution from \gls{rtklib} has a \acrfull{tow} value that is constant half a second delayed compared to the Piksi. This may be a result from the handling of \gls{tow} value from different satellites in \gls{rtklib}, such that a control system will not get a delayed position estimate. However for a integrated navigation system this pose a problem. The integration becomes more difficult when it's unclear how old the position estimate is.

The tracking performance from the two receiver type indicate that the Ublox is superior to the Piksi. It always manage to track more satellites, and keep track of them longer.
Before the take-off the Piksi had yet solved its integer ambiguity, while \gls{rtklib} had. This might be because it kept track of fever satellites. 
The flight test showed that keeping a fixed integer ambiguity solution during a flight is difficult, likely due to the dynamic behaviour of the \gls{uav}. Constraint on the behaviour such that the antenna is kept out of shadow zones will help, however the landing phase of the \gls{uav} operation must be kept independent from the rest of the operation. Therefore landing specific constraint cannot be imposed on the \gls{uav} general behaviour. Hence the navigation system must be able to recover its fixed integer ambiguity solution during the flight, which the \gls{rtklib} system proved the capability of.


%During the flight the Ublox lost track of more satellite then during the \gls{gps} lab tests. The antenna follows the orientation of the \gls{uav} and therefore some satellite will be blocked by the fuselage. This must be consider when designing a guidance system for automatic landing. To ensure that the antenna has a clear view of the sky, the \gls{uav} should try to have a maximum roll when the automatic landing system is engaged.

The velocity estimate from Piksi and \gls{rtklib} in the first test have similar estimate for the North and East component, but differ in the Down component. In the flight test the \gls{rtklib} altitude velocity was better, such that it can be used in a control system.
%
%
%
%
%
%
%
%DO NOT INCLUDE WHAT IS BELLOW BEFORE FURTHER WORK
%
%
%It might be that the float position estimate is good enough to perform a automatic landing. Further test are required.
%
%Rtklib managed to keep float/fixed solution during the landing.
%
%The rtklib was the only system able to get a fixed solution. Discus the performance of rtklib, whit the ublox receiver. 
%
%What can be concluded:
%
%Rtklib is able to faster find a fixed solution. Can be because of better receiver, better integer ambiguity resolution strategy, poorer validation of ambiguity, more efficient algorithm. Based on how many satellites the ublox receiver manage to track I will say that the receiver had a large part in shortening the time.
%
%When fixed there is not a large deviation between the two solutions. One difference that might cause problem in a guidance system is that it appear that the solution from rtklib is delayed. The \gls{tow} that's given with the solution do not include the millisecond value. 
%
%The Piksi is faster to calculate the position, but slower to resolve the integer ambiguity. 
%
%The conclusion of the different system. Piksi has better and more efficient solution software, however rtklib can use a superior \gls{gnss} receiver.
%
%Both system has a good estimation of the North and East velocity, however it appear that rtklib has problem estimating the vertical velocity.


\section{Further work}
Further work required for the development of an \gls{uav} system with automatic landing capabilities sufficient for landing in a net on a ship includes

Establishment of accuracy requirement by conduct over land automatic net landing.

Implementation of an integrated test system with automatic net landing capabilities.

Determine operation weather limitations, e.g wind, temperature and rain.

%The flight test shown that the navigation system has problem with keeping its fixed integer ambiguity solution during flight. Therefore further test are required to investigate the effect of mixed float and fixed solution during landing. This can be achieved by integrating the navigation system into a control and guidance system, which can be used to perform a trial landing with a safety margin to ensure that the \gls{uav} do not crash.
%
%Constraints on the behaviour of the \gls{uav} will help the navigation system to keep its fixed integer ambiguity solution during the landing. A upper limit on the roll and pitch angle that minimize the time the antenna is in a shadow zone will increase position accuracy from the navigation system.
%
%The Ublox receiver and the \gls{gnss} antennas are able to receive both \gls{gps} and \gls{glonass} L1 signals, which should be exploited to increase the number of valid satellite available for the \gls{rtk-gps} system. This could also give better constellation geometry, and help the system resolve its integer ambiguity faster.
% 
%
%A \gls{rtk-gps} system can use both the Piksi and \gls{rtklib} such that the automatic landing system has redundancy in position, and velocity estimation. A validation task must then be design to choose which of the system should send the rtkfix \gls{imc} message to the rest of the system.

\cleardoublepage