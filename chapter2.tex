%===================================== CHAP 2 =================================

\chapter{Coordinate systems}
This chapter will give an overview over different coordinate system and the relationship between them. A coordinate system is used to define a position relative to a origin. The choice of coordinate system is critical when describing the equation of motion. Newtons first law only apply when the origin is viewed as an inertial frame. When navigating on/or close to the surface there are two convension used; ENU and NED. Both the ENU and NED system is defines relative to the center of the Earth.

Write about the long lat system and the WGS 84, the ellipsoid and geosomething
\section{ECEF}
The Earth Centered, Earth Fixed coordinate system is defined in the center of the Earth with it's x-axis point toward the intersection between the Greenwich meridian and Equator ( $0\deg $ longitude, $0\deg $ latitude). The z-axis points along the Earth's rotational axis, and the y-axis complete the right handed orthogonal coordinate system. The ECEF system can be represented in either Cartesian coordinates (xyz) or ellipsoidal coordinates (longitude, latitude and height). 
The Earth centered, Earth Fixed system
\section{LLH}
\section{Ellipsoid}
How the the Earth ellipsoid defined. Advantages and disadvantages. What most be considered when estimating altitude. The difference between the ellipsoid and geoid. The geoid follows the curvature of the Earth. When flying perpenticular to the geoid there will be an angle between the normal from the geoid and the normal from the ellipsoid. 
\section{NED and ENU}
The North East Down frame and East North Up frame. See at lab assignment in gps.
Defined tangential to the Earth ellipsoid. The ellipsoid that is currently used is the WGS-84 ellipsoid ( ref til bok)
Different orientation
\cleardoublepage