HUSK AA SKRIVE HVA RMS STAAR FOR
\begin{table}[!h]
\begin{center}
    \begin{tabular}{ | l | l | l |}
    \hline
    \textbf{Source} & \textbf{Potential error size} & \textbf{Error mitigation and residual error} \\ \hline
     Satellite clock model & Clock modeling error: 2 m (rms) & DGPS: 0.0 m \\ \hline
     Satellite ephemeris prediction & Component of the ephemeris prediction error allong the line of sight: 2 m (rms) & DGPS: 0.1 m (rms) \\ \hline
     Ionospheric delay & Effect upon the code and the carrier is equal and opposite: The code is delayed while the carrier is advanced by the same amount.
     
     Delay in zenith direction $\approx 2-10 m $, depending upon user latitude, time of the day, and solar activity.
     
     Delay for a satellite at elevation angel $el = zenith delay \times obliquity factor$ (el)
     
     Obliquity factor: 1 at zenith; $1.8$ at $30\deg$ elevation angle; and 3 at $5\deg$. & DGPS = 0.2 m (rms) \\ \hline
     Troposheric delay & Code and the carrier are both delayed by the same amount. Delay in zenith direction at sea level $\approx 2.3-2.5 m$;lower at heigher altitudes Delay for satellite at elevation angel $el = zenith delay \times obliquity factor (el)$ Obliquity factor: 1 at zenith; 2 at $30\deg$ elevation angle; 4 at $15\deg$; and 10 at $5\deg$ \\ \hline
     Multipath & In a clean environment: $Code 0.5-1 m Carrier: 0.5-1 cm$ \\ \hline
     Receiver noise & Code:$0.25-0.5 m$ rms \\ \hline
    \end{tabular}
\end{center}
\caption{Table 1. }
\label{Tab1}
\end{table}