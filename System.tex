%===================================== CHAP 4 =================================

\chapter{System components and implementation}
This chapter will present the different software and hardware components used in this project work. The two first sections in this chapter will present the hardware and software components respectfully. This is followed by an overview over the components used in the \gls{uav} and base station that is used to create the navigation system. The last part of the chapter explains how the software and hardware components are implemented in the \gls{uav} test system.
\section{Hardware}
The different hardware components used in this work includes an \gls{uav},a payload computer, two \gls{gnss} receivers and two antennas.
\subsection{UAV}\label{ss:SkywalkerX8}
The \gls{uav} used is a Skywalker X8 from Skywalker Technology, see figure \ref{figure:skywalkerX8}, which is a fixed wing \acrfull{uav} that is moulded out of \gls{epo}, which makes it a cheap and robust platform for prototype testing and modifications. The large space within the fuselage makes it ideal for experimental payload and projects with modest requirements.
\begin{figure}[H]
	\centering
		\includegraphics[width=0.7\textwidth]{figs/Wing-X8_white-bgd2.png}
		\caption{X8 Skywalker from Skywalker Technologies, Picture from \url{www.campilot.tv/blog/win-x8}}
		\label{figure:skywalkerX8}
\end{figure}
The X8 has a wingspan of $2120mm$ which allow for a Maximum Take-Off Wight of $4.2kg$, including $1kg$ for the payload. The X8 used in this work is outfitted in according to the specification given in \citep{KlausenX8}. The components that is used in the X8 at the UAV-Lab is given in table \ref{tb:X8Components}.
\begin{table}[H]
\begin{center}
\begin{tabular}{l l}
Sensors & 3DR ublox GPS with Compass Kit\\&Pixhawk Airspeed Sensor Kit \\
Servo & Hitec HS-5125MG \\
Motor & Hacker A40 \\
ECU & Jeti Master Spin 66 Pro \\
Main Battery & 1 Zippy Compact 4S 5000 mAh\\
Autopilot & 3DR Pixhawk \\
Primary digital data link & Ubiquiti Rocket M5
\end{tabular}
\end{center}
\caption{Components in the X8 at the UAV-Lab}
\label{tb:X8Components}
\end{table}
\subsection{Embedded computer}
The embedded computer chosen as payload in the X8 is a Beaglebone Black \citep{BeagleBoneBlack}, shown in figure \ref{figure:BeagleBone}. The Beaglebone was chosen because of its sufficient computation power, low energy consumption and variety of communication interfaces. For ease interface access the  the UAV-Lab at NTNU has developed an extension board called "CAPE"\citep{KlausenX8}.
\begin{figure}[H]
	\centering
		\includegraphics[width=0.7\textwidth]{figs/BeagleBoneBlackE14.png}
		\caption{BeagleBone Black element 14, Picture from \url{http://www.element14.com}}
		\label{figure:BeagleBone}
\end{figure}
The Beaglebone black supports linux operating systems, and it is compatible with the \gls{lsts} toolchain, which is further presented in section \ref{S:software}.
\subsection{GNSS receiver}
The system will use two types of \gls{gnss} receivers, namely an Ublox LEA M8T receiver and a Piksi system receiver. 

The Ublox LEA M8T is a new generation of low-cost single frequency\gls{gnss} receiver from Ublox. The receiver support sending out raw \gls{gnss} data from both \gls{gps} and \gls{glonass}. The receiver have great performance in acquisition and tracking sensitivity of \gls{gnss} satellites. More technical details can be found in  \citep{UbloxDataSheet,UbloxReceiverDescription}.
\begin{figure}[H]
	\centering
		\includegraphics[width=0.7\textwidth]{figs/ubloxLeaM8T.jpg}
		\caption{Ublox LEA M8T, Picture from \url{http://www.csgshop.com}}
		\label{figure:Ublox}
\end{figure}
\subsubsection{Piksi}\label{ss:Piksi}
The Piksi system is a low cost, high performance \gls{gps} receiver with \gls{rtk-gps} functionality with capability for centimetre level relative positioning accuracy developed by Swift Navigation, shown in figure \ref{figure:Piksi}. Piksi is ideal for autonomous vehicles because of its small form factor, fast position solution update rate and low power consumption. 
More detailed information about Piksi can be found in the datasheet \citep{Piksiv231}
The communication structure for Piksi used in this project work is seen in figure \ref{figure:CommunicationPiksi}.
\begin{figure}[H]
\begin{subfigure}[H]{1\textwidth}
	\centering
		\includegraphics[width=0.7\textwidth]{figs/piksi_top.jpg}
		\caption{Piksi receiver, Picture from \url{www.swiftnav.com}}
		\label{figure:Piksi}
\end{subfigure}

\begin{subfigure}[H]{1\textwidth}
	\centering
		\includegraphics[width=0.8\textwidth]{figs/Piksi.png}
		\caption{Communication structure for Piksi}
		\label{figure:CommunicationPiksi}
\end{subfigure}		
\end{figure}

\subsection{GNSS antenna}
The navigation system uses two \gls{gnss} antenna, one for the \gls{uav} and the other for the base station. The main criteria for the \gls{uav} antenna is that it's small,compact and have a light weight.
The selected antenna type M1227HTC-A-SMA, seen in figure \ref{figure:Maxtena}, which has been used in other \gls{uav} set-ups at the UAV-Lab with good results. The antenna is small, compact and with an light weight of $17g$. It's design for L1/L2 gps/glonass bands. Further information or specification can be found in the datasheet \citep{maxtena}

The base station do not have any restriction on size, weight or aerodynamic. The important factor for a base station antenna is that it has good multipath rejection. Also the base station position should be calculated as accurate as possible which impose further restriction on interference handling, phase center stability and noise rejection.
The antenna Novatel GPS-701-GG, seen i figure \ref{figure:Novatel}, was chosen as the base station antenna. The antenna has excellent multipath rejection with a highly stable phase center. It has reception for both \gls{gps} and \gls{glonass} L1 signals.Further information or specification can be found in the datasheet \citep{novatel}.

\begin{figure}[H]
	\centering
		\includegraphics[width=0.5\textwidth]{figs/Maxtena-M1227HCT-A-SMA-image.jpg}
		\caption{The rover \gls{gnss} antenna, Picture from \url{http://sigma.octopart.com/21411362/image/Maxtena-M1227HCT-A-SMA.jpg}}
		\label{figure:Maxtena}
\end{figure}

\begin{figure}[H]
	\centering
		\includegraphics[width=0.5\textwidth]{figs/702-L.png}
		\caption{The base station \gls{gnss} antenna, Picture from \url{http://www.novatel.com}}
		\label{figure:Novatel}
\end{figure}


\section{Software}\label{S:software}
This section contain the different software that is used in the X8 system. The following sections contain the operating system that runs on the base station and rover, the runtime environment used to perform the different tasks, the messages protocol, the missionplaner program, and the navigation system used, e.g. \gls{rtklib} and Piksi.
\subsection{LSTS toolchain}
The Laboratório de Sistemas e Tecnologia Subaquática (LSTS) toolchain is used as a platform for software implementation. This is a flexible, scalable, open-source software that supports integration and control of various types of unmanned vehicles \citep{pinto2013lsts}. The toolchain includes a operating system, a runtime environment, a message protocol and a command and control for operations. The following program are included in the toolchain
\subsubsection{Glued}
Glued is a minimal Linux operating system distribution, and design with embedded system in mind. It is platform independent, easy to configure and contain only the necessary packages to run on a embedded system. This makes GLUED a light and fast distribution. GLUED is configured through a single configuration file that which can be created for a specific system. 
\subsubsection{Dune}
Dune (DUNE Uniform Navigation Environment) is a runtime environment for unmanned systems. DUNE is operating system independent and can run on-board the embedded computer. It can interact with the sensors, actuators and payload. In addition it can be used for tasks like communication, navigation, control, manoeuvring, plan execution and supervision.
Dune works by setting up individual task that can dispatch and subscribe to different \gls{imc} messages.
\subsubsection{IMC}\label{ss:IMC}
The \acrfull{imc} protocol is build to interconnect systems of vehicles,sensors and human operators. The protocol is a messages-oriented protocol that enable exchange of real-time information about the environment and updated objectives, such that the participant in the communication can pursue a common goal cooperatively.
\gls{imc} has a standard way of dispatching and consuming messages, which abstracts hardware configuration from the software. All \gls{imc} messages are logged by DUNE.

%Other feature: Not assuming a specific software architecture, can generate native support for different programming languages and/or computer architectures. Used in network nodes, inter-process and inter thread communication
\subsubsection{Neptus}
Neptus is an open source command and control software that can be used for a single or fleet of unmanned vehicles with different types of sensors. The operator can observe real time data from a vehicle, review previous missions and plan future mission. In this work Neptus has been used to extract data logged by Dune during a defined mission, and to monitor the integer ambiguity solution from \gls{rtklib} and Piksi.
\subsection{RTKLIB}\label{ss:Rtklib}
\acrfull{rtklib}\citep{takasu2009development} is a open source program package for standard and precise positioning with GNSS developed by T. Takasu. \gls{rtklib} can be configured to apply \gls{rtk-gps}, such that raw GNSS data is used estimate the position of the rover in real time. Figure \ref{figure:RTKLIB_STRUCTURE} shows how \gls{rtklib} can be used in a \gls{rtk-gps} mode. The two main modules here is str2str and rtkrcv. Both will be explained more closely in the following sections. More information about  \gls{rtklib} can be found in the manual \citep{Rtklib242}.

\begin{figure}[H]
	\centering
		\includegraphics[width=1\textwidth]{figs/RTKLIB.png}
		\caption{The communication structure of \gls{rtklib}}
		\label{figure:RTKLIB_STRUCTURE}
\end{figure}
\subsubsection{Rtkrcv}
As part of the \gls{rtklib} Rtkrcv is  used to calculate the position of the rover in real time.
Rtkrcv can be configured to have two output streams. It's desired in a automatic landing system to have a velocity estimate. However this is not provided in the newest version of \gls{rtklib}, and therefore the source code had to be altered to send out the velocity data. The position output is in \gls{enu} format and the full output structure is presented in table \ref{Tb:RtklibOutput}
\begin{table}[H]
\begin{center}
    \begin{tabular}{ | l | l |}
    \hline
    \textbf{Header} & \textbf{Content} \\ \hline
     1 Time & The epoch time of the solution indicate the true receiver\\& signal reception time. Can have the following format:\\&\\& yyyy/mm/dd HH:MM:SS.SSS:\\& Calender time in GPST, UTC or JST.\\&\\&
     
     WWWW SSSSSSS.SSS:\\&
     GPS week and TOW in seconds  \\ \hline
     2 Receiver Position & The rover receive antenna position \\ \hline
     3 Quality flag (Q) & The flag which indicates the solution quality.\\& 1:Fixed\\& 2:Float\\& 5:Single \\ \hline
     4 Number of valid satellites (ns) & The number of valid satellites for solution estimation. \\ \hline
     5 Standard deviation & The estimated standard deviation of the\\& solution assuming a priori error model and error\\& parameters by the positioning options \\ \hline
     6 Age of differential & The time difference between the observation data epochs\\& of the rover receiver and base station in second. \\ \hline
     7 Ratio factor & The ratio factor of "ratio-test" for standard integer\\& ambiguity validation strategy \\ \hline
     8 Receiver velocity & The velocity of the rover. Given only when output is\\& in enu format \\ \hline
    \end{tabular}
\end{center}
\caption{Rtklib output solution format }
\label{Tb:RtklibOutput}
\end{table}

When configured as a \gls{rtk-gps} the rtkrcv must resolves the integer ambiguity, which relate to lack of information on the number of whole phase cycles between the receiver and a satellite. In rtkrcv this is managed by implementation of the \gls{lambda} method, which is explain in \citep{LAMBDAMETHOD,Ambiguity:Estimation,LAMBDA:METHOD}.
Further the position solution is calculated using a Extended Kalman Filter, where the structure of the filter is depending on the configuration of rtkrcv.
\subsubsection{Str2str}
Str2str is used as a base station program that can receive raw \gls{gnss} data and further export data over tcp server, set-up by str2str. Str2str sends out Radio Technical Commission for Maritime Service 3 (RMTC3) formatted messages, however it can be configured to send whatever comes in as input. The communication between str2str and rtkrcv is shown i figure \ref{figure:RTKLIB_STRUCTURE}
\section{Implementation}
This section explain how the \gls{rtk-gps} navigation system has been implemented in the \gls{uav} system used in the testing. The final implementation includes both use of \gls{rtklib} and Piksi for position estimation, as shown in figure \ref{figure:HardSoft}


\subsection{Software implementation}
The software implementation is shown in figure \ref{figure:HardSoft}, where the different modules are available from the UAV-Lab. The detailed implementation has been trough configuration of the config files, in addition to alteration of the existing implementation.
\subsubsection{Navigation system}
The \gls{rtk-gps} module in the navigation system includes \gls{rtklib}, and the two Dune tasks RTKGPS and Piksi. The RTKGPS task is connected to \gls{rtklib} though a virtual connection, and the Piksi task has a physical connection to the Piksi receiver, as shown i figure \ref{figure:HardSoft}.
\begin{figure}[H]
	\centering
		\includegraphics[width=1\textwidth]{figs/Combined.png}
		\caption{System implementation}
		\label{figure:HardSoft}
\end{figure}

\gls{rtklib} is implemented into the base station and the rover. The base station implementation use the  str2str program to communicate with the Ublox over a uart cable, outputs raw \gls{gnss} data over tcp to the rover.
The rover uses the rtkrcv program from \gls{rtklib} to estimate the position of the rover. Rtkrcv receive raw \gls{gnss} data from both the str2str program and the Ublox installed in the \gls{uav}, as shown in figure \ref{figure:RTKLIB_STRUCTURE}. Rtkrcv is configured in a moving baseline configuration to simulate the behaviour that is expected during a landing on a ship. The configuration file is included in \ref{APPENDIX:RTKLIB}.

The output from the RTKGPS task is a \gls{imc} message called RtkFix, see table \ref{Tb:RtkFix}, which include the relative position of the \gls{uav} as well as the velocity, type of integer solution and the \gls{gps} \acrfull{tow}. The \gls{imc} message is further sent to an external computer for monitoring over TCP/IP (Wifi).
\begin{table}[H]
\begin{center}
    \begin{tabular}{ | l | l |}
    \hline
    \textbf{Header} & \textbf{Content} \\ \hline
     tow & Gps time of Week  \\ \hline
     n & Baseline North coordinate \\ \hline
     e & Baseline East coordinate \\ \hline
     d & Baseline Down coordinate \\ \hline
     v\verb=_=n & Velocity North coordinate \\ \hline
     v\verb=_=e & Velocity East coordinate \\ \hline
     v\verb=_=d & Velocity Down coordinate \\ \hline
     iar\verb=_=hyp & Number of hypotheses in the Integer Ambiguity Resolution \\ \hline
     iar\verb=_=ratio & Quality ratio of Integer Ambiguity Resolution \\ \hline
     type & Type of fix: \\& None = No solution, but RTK task is running
     \\& Obs = No solution, but receiving observations
     \\& Float = Floating point solution of Integer Ambiguity Resolution
     \\& Fix = Fixed(single) solution of Integer Ambiguity Resolution \\ \hline
    \end{tabular}
\end{center}
\caption{The \gls{imc} message RtkFix }
\label{Tb:RtkFix}
\end{table}

\subsection{Hardware implementation}
Both the base station and \gls{uav} has been fitted with a \gls{gps} antenna splitter, seen i figure \ref{figure:AntennaSplitter}, such that both receivers receive the same signals from the antennas. GLUED is used as the operating system in the Beaglebone on both the base station and the \gls{uav}. The Piksi and Ublox is connected with the Beaglebone over uart cables. The primary data-link between the \gls{uav} and the base station with Ubiquiti AirMax radios.
The embedded computer uses GLUED as its operating system, and on it runs both Dune and \gls{rtklib}. The Piksi and Ublox is connected to the BeagleBone over uart cables.
More information on the hardware setup used in the X8 and the Base station is given in \citep{KlausenX8}
\begin{figure}[H]
	\centering
		\includegraphics[width=0.7\textwidth]{figs/066.jpg}
		\caption{\gls{gps} antenna splitter}
		\label{figure:AntennaSplitter}
\end{figure}
\cleardoublepage