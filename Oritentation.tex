%===================================== CHAP 2 =================================

\chapter{Reference frames}
This chapter will give an overview over different coordinate system and the relationship between them. A coordinate system is used to define a position relative to a origin. The choice of coordinate system is critical when describing the equation of motion. Newtons first law only apply when the origin is viewed as an inertial frame. When navigating on/or close to the surface there are two convension used; ENU and NED. Both the ENU and NED system is defines relative to the center of the Earth.

Write about the long lat system and the WGS 84, the ellipsoid and geosomething
\section{ECI}
\gls{eci} frame is considered a inertial frame for terrestrial navigation. The origin is fixed in the center of the Earth, and the axis is fixed in space.
\section{ECEF}
The \gls{ecef} coordinate system is defined in the center of the Earth with it's x-axis point toward the intersection between the Greenwich meridian and Equator ( $0\deg $ longitude, $0\deg $ latitude). The z-axis points along the Earth's rotational axis, and the y-axis complete the right handed orthogonal coordinate system. The \gls{ecef} system can be represented in either Cartesian coordinates (xyz) or ellipsoidal coordinates (longitude, latitude and height). The \gls{ecef} frame rotate relative to the \gls{eci} frame at a angular rate of $\omega_e = 7.2921 \times 10^{-5}rad/s$, where $\omega_e$ is the Earth rotation.
\section{NED and ENU}
The \gls{ned} and \gls{enu} frame is defined as relative to the Earth reference ellipsoid (\gls{wgs-84}). For the \gls{ned} frame the x axis points in the direction to the true North, y axis towards East while the z axis points downward to completed the right handed orthogonal coordinate system. The \gls{enu} has the x and y axis exchange place in respect of the \gls{ned} frame, and the z axis point upwards instead of downwards.
\cleardoublepage