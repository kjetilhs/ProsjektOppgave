%===================================== CHAP 1 =================================

\chapter{Introduction}
This chapter the background and motivation to why this project thesis has been written. Also reference to privios work, the goal of the thesis and the layout.
\section{Background and motivation}
Why is the project thesis relevant: Installing better GPS to enable better hight estimation

What have I done? Integrated new GPS with DUNE

What is next? Land the UAV. Test algorthm to Frølich, better state estimation 

Unmanned aerial vehicles have in seen an increase usage in the civilian sector on land, where they can be used at a cheaper price then manned aircraft. However this has not yet been the case in the civil maritime sector, due to a harsher environment and smaller landing areas.

A UAV can increase performance in many maritime operation where today only other manned aircraft or satellites are the only solution. In the maritime sector they can be used in iceberg management, monitoring of oil spills, search and rescue and maritime traffic monitoring. To enable a safe UAV operation at sea there must be a system in place to ensure a safe landing.

There exist landing system that can guide the UAV towards a net, but they are expensive and restricted to a few UAVs. A pilot could always land the UAV, but it would be better if the UAV could hit the net by it self. In order to make the UAV able to perform a automatic landing the minimum requirement is that it know where it is at all time. This put a requirement of the position sensor. The position system need to combine low cost with high position accuracy. This thesis will test a new generation of GNSS receiver, and use GPS to find the position to the UAV. The demand on the accuracy is that the error must be in decimetres to ensure safe landing in the net.


\section{Previous work}
The work done by Frølich on simulation of a net landing. Work done by Spockeli. Need other research field. GPS navigation is a well researched field. Relevant to this task is landing with gps, or high accurate position estimation. 

Other master thesis, paper on visual aid landing system. Frølich did the same research, using reasearch from Spocli.
\section{Goal of thesis}
Table \ref{Tb:GoalList} summaries the goal with this project thesis.
\begin{itemize}\label{Tb:GoalList}
\item Install the Ublox Lea M8T on the base station and x8
\item Configure rtklib to calculate the relative position of the x8 in relation to the base station
\item Make Dune able to receive the output stream from rtklib, and support a new imc message that contain the relative position of the x8
\item Test and compare the Ublox receiver against the pixi(TRENGER MODEL NAVN) receiver
\end{itemize}
Install new GPS receiver on the x8 and base station. Test and compare to the pixi gps that is the standard solution i the uavlab. Evaluate if the new gps is good enough to enable automatic landing
Install new GPS reciver with higher accuracy to enable automatic landing
\section{Layout of thesis}
\begin{verbatim}
\begin{eqnarray}\label{eq1}
F = m \times a
\end{eqnarray}
\end{verbatim}

\noindent This will produce

\begin{eqnarray}\label{eq1}
F = m \times a
\end{eqnarray}

\noindent To refer to the equation

\begin{verbatim}
\eqref{eq1}
\end{verbatim}

\noindent This will produce \eqref{eq1}.


\section{Figures}
To create a figure

\begin{verbatim}
\begin{figure}[h!]
  \centering
    \includegraphics[width=0.5\textwidth]{fig/pikachu}
  \caption{Pikachu.}
\label{fig1}
\end{figure}
\end{verbatim}



\noindent To refer to the figure

\begin{verbatim}
\textbf{Fig. \ref{fig1}}
\end{verbatim}

\noindent This will produce \textbf{Fig. \ref{fig1}}

\section{References}

To cite references

\begin{verbatim}
\cite{1,2,3}
\end{verbatim}
or
\begin{verbatim}
\citep{1,2,3}
\end{verbatim}

\noindent This will produce: \cite{1,2,3} or \citep{1,2,3}, respectively.

\section{Tables}

To creat a table

\begin{verbatim}
\begin{table}[!h]
\begin{center}
    \begin{tabular}{ | l | l | l | l |}
    \hline
    \textbf{No.} & \textbf{Data 1} & \textbf{Data 2} \\ \hline
     1 & a1 & b1 \\ \hline
     2 & a2 & b2 \\ \hline
    \end{tabular}
\end{center}
\caption{Table 1.}
\label{Tab1}
\end{table}
\end{verbatim}

\noindent This will produce

\begin{table}[!h]
\begin{center}
    \begin{tabular}{ | l | l | l | l |}
    \hline
    \textbf{No.} & \textbf{Data 1} & \textbf{Data 2} \\ \hline
     1 & a1 & b1 \\ \hline
     2 & a2 & b2 \\ \hline
    \end{tabular}
\end{center}
\caption{Table 1.}
\label{Tab1}
\end{table}

\noindent To refer to the table

\begin{verbatim}
\textbf{Table. \ref{Tab1}}
\end{verbatim}

\noindent This will produce \textbf{Table. \ref{Tab1}}.

\cleardoublepage