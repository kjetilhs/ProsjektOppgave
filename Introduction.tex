%===================================== CHAP 1 =================================

\chapter{Introduction}

\section{Background and motivation}
In the resent past the development of flying \glspl{uav} have seen to provide an attractive alternative to task previously performed by manned alternatives. Typical task which has attracted attention includes inspection, aerial photography, environmental surveillance and search and rescue. Today \gls{uav} are mostly operated over land, but in the future this will include the out at sea as well. This will give some challenges that must be overcome. One of these challenges is that the \gls{uav} should be able to perform a autonomous landing.

A \gls{uav} can provide an attractive alternative for many maritime operation where today only other manned aircraft or satellites are the only solution. In the maritime sector they can be used in iceberg management, monitoring of oil spills, search and rescue and maritime traffic monitoring.

An obvious premise for successfully and safe \gls{uav} operation, in particular at sea, is the provision of a robust system for sage ladning of the \gls{uav} following a complete mission. In order to complete an automatic landing a path planner, guidance system and a accurate position estimation system, in addition to the low level control system in the \gls{uav} is required. Such complete system is still subject for development, where only method for achieving the different objectives has been developed, i.e. path planner and guidance system respectively. The path planer and guidance system as basis for this work was developed as previous master thesis work in the UAVLAB by \citep{Froelich}. The gudiance system is currently under further development in two project thesis. A key component in the guidance system is the position estimation system, which for the guidance referenced was developed by \citep{Spockeli}. However, this system has been concluded to be in-sufficient with respect to provide means required for automatic landing.

There exist landing system that can guide the \gls{uav} towards a net, but they are expensive and restricted to a few \glspl{uav}. A pilot could always land the \gls{uav}, but it would be better if the \gls{uav} could hit the net by it self. In order to make the \gls{uav} able to perform a automatic landing the minimum requirement is that it know where it is at all time. This require a accurate position estimation in real time. A highly accurate position sensor is typically expensive, however it's possible to achieve a accurate solution with low cost sensors. By combining two \gls{gnss} receivers it's possible to estimate the relative position of one of the receivers in respect of the other receiver highly accurately. This work will test a new generation of \gls{gnss} receiver, and use \gls{gps} to find the relative position of the \gls{uav}. The demand on the accuracy is that the error must be in decimetres to ensure safe landing in the net.

This work will continue the research done by Spockeli, and introduce a new \gls{gnss} receiver that will be used with the open source program, rtklib. The motivation is to have a system with accurate local position estimate, such that in the future a landing can be performed automatic.

The automatic landing system will use \gls{rtk-gps} for position estimation. The main motivation for this work is to describe the gaps of the available position system sufficient to scope further work required for closure of such gaps ultimately providing means for position estimating sufficient for completion of automatic landing.
\section{Literature Review}
Automatic landing in a net has privously been successful performed in the NTNU MSc thesis \citep{Skulstad&Syversen}. They managed to land a \gls{uav} in a stationary net using \gls{rtk-gps}. A other successfull automatic landing was done in the Stellenbosch University MSc thesis \citep{smit2013autonomous} using \acrfull{dgps}. That thesis focused more on the control design.

Further research on the automatic landing system at the UAVLAB at NTNU was done in the MSc thesis by \citep{Spockeli} and \citep{Froelich}. In the MSc thesis by Spockeli he studied a integrated \gls{rtk-gps}/\gls{ins} navigation system with a nonlinear observer. In the MSc thesis by Froelich he create a control and guidance system that in simulation has successfully managed to follow a landing trajectory. He also created a dynamical path generation system that allows the net to move, and re-plan the landing path.

In the paper \citep{kim2013fully} it was proposed a vision based landing system, that would detect the recovery net, and plan a landing path. The system was successfully tested and is a valid alternative for a low-cost autonomous landing system.

There has been done work on autonomous landing system using a vision aided system. The work done in \citep{williams2012intelligent} describe a intelligent vision aided landing system that detect and generate landing waypoints for a unsurveyed airfield.

Discussion on the carrier phase ambiguity resolution was done in \citep{GeodeticBaselines}.  A well used strategy used to resolve the integer ambiguity was proposed in \citep{Ambiguity:Estimation}, and further disused in \citep{LAMBDA:METHOD,LAMBDAMETHOD}. 

Work on precise 3D positioning of \gls{uav} done in \citep{3D-RTK}. Tested the capabllity of rtklib

Uses rtkgps to take photo
\citep{Low-costRTK}  



%Commercial landing system:  \citep{scanealge}
\section{Scope of work}
The scope of work is to validate the performance of suitable positioning systems by study and testing, and to identify gaps required to be closed for successful implementation for a integrated autonomous Guidance, Navigation and Control system which will allow for automatic landing of a \gls{uav} at sea. This will include:
\begin{table}[!h]
\begin{center}
    \begin{tabular}{ l}
     Testing of the performance of Ublox LEA M8T  \\ 
     Compare the performance of \gls{rtklib} and Piksi \\
     Compare the real time estimate with the post processed estimate 
    \end{tabular}
\end{center}
\label{Tb:Evasion}
\end{table}
The scope of this work is to test the performance of ublox-LEA M8T \gls{gnss} receiver in a real time differential \gls{dgps} configuration. The \gls{rtk-gps} solution will be calculated with the open source program rtklib, which will communicate with a task in DUNE. The solution from rtklib will be compared against the solution from Piksi, and a post processed solution from rtklib. The result from the experiment will be used in the discussion on how to perform a automatic landing.

%For validation of the position accuracy needed the evasion criteria given in \ref{Tb:Evasion} \citep{Froelich} will be used.
%\begin{table}[!h]
%\begin{center}
%    \begin{tabular}{ | l | l |}
%    \hline
%    \textbf{Criteria} & \textbf{Value} \\ \hline
%     Cross-track error & $\pm1$ meter  \\ \hline
%     Altitude error & $\pm1$ meter \\ \hline
%    \end{tabular}
%\end{center}
%\caption{Evasion criteria }
%\label{Tb:Evasion}
%\end{table}

\section{Layout}
This section is currently not up to date

Chapter 1 Intro

Chapter 2 Theory about coordinate system

Chapter 3 Theory about GNSS system with focus on the GPS

Chapter 4 Hardware and software

Chapter 5 Test and result

Chapter 6 Conclusion, discussion and further work


%EVERYTHING BELLOW BEFORE CHAPTER 2 IS NOT PART OF THE THESIS.
%\begin{verbatim}
%\begin{eqnarray}\label{eq1}
%F = m \times a
%\end{eqnarray}
%\end{verbatim}
%
%\noindent This will produce
%
%\begin{eqnarray}\label{eq1}
%F = m \times a
%\end{eqnarray}
%
%\noindent To refer to the equation
%
%\begin{verbatim}
%\eqref{eq1}
%\end{verbatim}
%
%\noindent This will produce \eqref{eq1}.
%
%
%\section{Figures}
%To create a figure
%
%\begin{verbatim}
%\begin{figure}[h!]
%  \centering
%    \includegraphics[width=0.5\textwidth]{fig/pikachu}
%  \caption{Pikachu.}
%\label{fig1}
%\end{figure}
%\end{verbatim}
%
%
%
%\noindent To refer to the figure
%
%\begin{verbatim}
%\textbf{Fig. \ref{fig1}}
%\end{verbatim}
%
%\noindent This will produce \textbf{Fig. \ref{fig1}}
%
%\section{References}
%
%To cite references
%
%\begin{verbatim}
%\cite{1,2,3}
%\end{verbatim}
%or
%\begin{verbatim}
%\citep{1,2,3}
%\end{verbatim}
%
%\noindent This will produce: \cite{1,2,3} or \citep{1,2,3}, respectively.
%
%\section{Tables}
%
%To creat a table
%
%\begin{verbatim}
%\begin{table}[!h]
%\begin{center}
%    \begin{tabular}{ | l | l | l | l |}
%    \hline
%    \textbf{No.} & \textbf{Data 1} & \textbf{Data 2} \\ \hline
%     1 & a1 & b1 \\ \hline
%     2 & a2 & b2 \\ \hline
%    \end{tabular}
%\end{center}
%\caption{Table 1.}
%\label{Tab1}
%\end{table}
%\end{verbatim}
%
%\noindent This will produce
%
%\begin{table}[!h]
%\begin{center}
%    \begin{tabular}{ | l | l | l | l |}
%    \hline
%    \textbf{No.} & \textbf{Data 1} & \textbf{Data 2} \\ \hline
%     1 & a1 & b1 \\ \hline
%     2 & a2 & b2 \\ \hline
%    \end{tabular}
%\end{center}
%\caption{Table 1.}
%\label{Tab1}
%\end{table}
%
%\noindent To refer to the table
%
%\begin{verbatim}
%\textbf{Table. \ref{Tab1}}
%\end{verbatim}
%
%\noindent This will produce \textbf{Table. \ref{Tab1}}.

\cleardoublepage