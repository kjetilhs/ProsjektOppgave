%===================================== CHAP 1 =================================

\chapter{Introduction}

\section{Background and motivation}
Recently development of flying \glspl{uav} have been recognized to provide an attractive alternative to work previously performed by manned operations. Typical work which has attracted attention includes inspection, aerial photography, environmental surveillance and search and rescue. Today \glspl{uav} are mostly operated over land, however in the future this will include over sea as well. This will give some challenges which must be overcome. One of these challenges is that the \gls{uav} need to be able to perform a autonomous landing.

An \gls{uav} can provide an attractive alternative for many maritime operation where today manned aircraft or satellites is the only solution. In the maritime sector \gls{uav} can be used in iceberg management, monitoring of oil spills, search and rescue and maritime traffic monitoring.

An important premise for successful and safe \gls{uav} operation, in particular at sea, is the provision of a robust system for safe landing of the \gls{uav} on a vessel following completed operations. In order to perform an automatic landing a path planner, guidance system and accurate position estimation system are required, in addition to the low level control system in the \gls{uav}. Such complete system is subject for further development. Currently only method for achieving the individual objectives has been developed, i.e. path planner and guidance system respectively.

The path planer and guidance system as basis for this work were developed as previous master thesis work in the UAVLAB by \citep{Froelich}. A key component in the guidance system is the position estimation system, which for the guidance referenced was developed by \citep{Spockeli}. However, this system has been concluded to be in-sufficient with respect to provide means required for automatic landing.

Existing landing system can guide the \gls{uav} towards a net, but they are expensive and restricted to a few \glspl{uav}. A pilot can land the \gls{uav}, however a better alternative would be if the \gls{uav} could land it self by the use of a net. In order to make the \gls{uav} able to perform a automatic landing a minimum requirement is that it knows its position at any time. This will require an accurate position estimation in real time. A highly accurate position sensor is expensive, however it's possible to achieve accurate solution with low cost sensors. This can be done by combining two \gls{gnss} receivers to estimate the relative position of one of the receivers in respect of the other receiver highly accurately. This project work will test a new generation of \gls{gnss} receiver, and use \gls{gps} to establish the relative position of the \gls{uav}. The navigation system must be accurate enough to correctly estimate if the \gls{uav} is following the generated landing path or if the deviation from the path is large enough such that an evasion manoeuvre is required. The position evasion criteria used in \citep{Froelich} is $\pm1m$ cross-track error and $\pm1m$ altitude error.


This project work will continue the research done by Spockeli \citep{Spockeli}, and use a new \gls{gnss} receiver that will be used with the open source program, \gls{rtklib}. The goal is to establish a system with accurate local position estimate, such that in the future a \gls{uav} net landing can be performed automatic.

The automatic net landing system will use \gls{rtk-gps} for position estimation. The main goal for this work is to describe the gaps of the available position system sufficient to scope further work required for closure of such gaps ultimately providing means for position estimating sufficient for completion of automatic landing.
\section{Literature Review}
Automatic landing in a net has privously been successful performed in the NTNU MSc thesis \citep{Skulstad&Syversen}. They managed to land a \gls{uav} in a stationary net using \gls{rtk-gps}. Unfortunately a different hardware and software setup at the Uavlab has made the code they used obsolete.

Further research on the automatic landing system at the UAVLAB at NTNU was done in the MSc thesis by \citep{Spockeli} and \citep{Froelich}. In the MSc thesis by Spockeli he studied a integrated \gls{rtk-gps}/\gls{ins} navigation system with a nonlinear observer. In the MSc thesis by Froelich he create a control and guidance system that in simulation has successfully managed to follow a landing trajectory. He also created a dynamical path generation system that allows the net to move, and re-plan the landing path. The control and guidance system created by Froelich has yet to be integrated with the work done by Spockeli. Froelich never did a physical test of the landing system, and the position estimate used in the system was not based on \acrfull{rtk-gps}.

 An other successful automatic landing was done in the Stellenbosch University MSc thesis \citep{smit2013autonomous} using \acrfull{dgps}. This MSc thesis gives a description on the control system required to perform an automatic landing, however the system in the thesis require a runway in order to land.
 
An other low-cost system that can be used for automatic landing that is vision based landing system. In the paper \citep{kim2013fully} it was proposed a vision based landing system, that would detect the recovery net, and plan a landing path. The system was successfully tested and is a valid alternative for a low-cost autonomous landing system. An other vision based landing system was proposed in \citep{williams2012intelligent}. This paper describes an intelligent vision aided landing system that can detect and generate landing waypoints for a unsurveyed airfield. The drawback with a vision based landing system is that it require much computational power. In addition the visual line of sight can quickly decrease during a operation.

\acrfull{rtk-gps} apply phase measurement of the \gls{gnss} signal for position estimation. In order to get a accurate position estimate the integer ambiguity must be resolved. An integer ambiguity resolution strategy was proposed in \citep{GeodeticBaselines}, and demonstrated centimeter level accuracy for a baseline up to $2000km$. Further studies on integer ambiguity resolution strategies resolved in the \acrfull{lambda} strategy, which was proposed in \citep{Ambiguity:Estimation}, and further disused in \citep{LAMBDA:METHOD,LAMBDAMETHOD}. The \gls{lambda} has been wildly used, and has proven a quick strategy to resolve the integer ambiguity, which makes it ideal for \gls{rtk-gps} systems. 

The use of \gls{rtk-gps} for accurate position estimation has been studied in \citep{3D-RTK}. The paper proposed how to create a low-cost \gls{rtk-gps} system that can accurate measure the position in 3D. They used raw \gls{gnss} data from the \gls{gnss} receiver and the program library \acrfull{rtklib} to estimate the position of a trolley in real time.

In the paper \citep{Low-costRTK} it was studied high precision positioning of micro-sized \gls{uav} using \gls{rtk-gps}. The system used a \gls{gnss} receiver that used a ground based augmentation system as a base station. The use of a ground based augmentation system us advantageous if the \gls{uav} can communicate with the ground station. For \gls{uav} operations where information from a ground based augmentation system is not available a local reference station must be considered.



%Commercial landing system:  \citep{scanealge}
\section{Scope of work}
The scope of work is to validate the performance of suitable positioning systems by study and testing, and to identify gaps required to be closed for successful implementation for a integrated autonomous Guidance, Navigation and Control system which will allow for automatic landing of a \gls{uav} at sea. This will include:
\begin{table}[!h]
\begin{center}
    \begin{tabular}{ l}
     Testing of the performance of Ublox LEA M8T  \\ 
     Compare the performance of \gls{rtklib} and Piksi \\
     Compare the real time estimate with the post processed estimate 
    \end{tabular}
\end{center}
\label{Tb:Evasion}
\end{table}
The scope of this work is to test the performance of ublox-LEA M8T \gls{gnss} receiver in a real time differential \gls{dgps} configuration. The \gls{rtk-gps} solution will be calculated with the open source program \gls{rtklib}, which will communicate with a task in DUNE. The solution from \gls{rtklib} will be compared against the solution from Piksi, and a post processed solution from \gls{rtklib}. The result from the experiment will be used in the discussion on how to perform an automatic net landing.

%For validation of the position accuracy needed the evasion criteria given in \ref{Tb:Evasion} \citep{Froelich} will be used.
%\begin{table}[!h]
%\begin{center}
%    \begin{tabular}{ | l | l |}
%    \hline
%    \textbf{Criteria} & \textbf{Value} \\ \hline
%     Cross-track error & $\pm1$ meter  \\ \hline
%     Altitude error & $\pm1$ meter \\ \hline
%    \end{tabular}
%\end{center}
%\caption{Evasion criteria }
%\label{Tb:Evasion}
%\end{table}

\section{Layout}
This section is currently not up to date

Chapter 1 Intro

Chapter 2 Theory about coordinate system

Chapter 3 Theory about GNSS system with focus on the GPS

Chapter 4 Hardware and software

Chapter 5 Test and result

Chapter 6 Conclusion, discussion and further work


%EVERYTHING BELLOW BEFORE CHAPTER 2 IS NOT PART OF THE THESIS.
%\begin{verbatim}
%\begin{eqnarray}\label{eq1}
%F = m \times a
%\end{eqnarray}
%\end{verbatim}
%
%\noindent This will produce
%
%\begin{eqnarray}\label{eq1}
%F = m \times a
%\end{eqnarray}
%
%\noindent To refer to the equation
%
%\begin{verbatim}
%\eqref{eq1}
%\end{verbatim}
%
%\noindent This will produce \eqref{eq1}.
%
%
%\section{Figures}
%To create a figure
%
%\begin{verbatim}
%\begin{figure}[h!]
%  \centering
%    \includegraphics[width=0.5\textwidth]{fig/pikachu}
%  \caption{Pikachu.}
%\label{fig1}
%\end{figure}
%\end{verbatim}
%
%
%
%\noindent To refer to the figure
%
%\begin{verbatim}
%\textbf{Fig. \ref{fig1}}
%\end{verbatim}
%
%\noindent This will produce \textbf{Fig. \ref{fig1}}
%
%\section{References}
%
%To cite references
%
%\begin{verbatim}
%\cite{1,2,3}
%\end{verbatim}
%or
%\begin{verbatim}
%\citep{1,2,3}
%\end{verbatim}
%
%\noindent This will produce: \cite{1,2,3} or \citep{1,2,3}, respectively.
%
%\section{Tables}
%
%To creat a table
%
%\begin{verbatim}
%\begin{table}[!h]
%\begin{center}
%    \begin{tabular}{ | l | l | l | l |}
%    \hline
%    \textbf{No.} & \textbf{Data 1} & \textbf{Data 2} \\ \hline
%     1 & a1 & b1 \\ \hline
%     2 & a2 & b2 \\ \hline
%    \end{tabular}
%\end{center}
%\caption{Table 1.}
%\label{Tab1}
%\end{table}
%\end{verbatim}
%
%\noindent This will produce
%
%\begin{table}[!h]
%\begin{center}
%    \begin{tabular}{ | l | l | l | l |}
%    \hline
%    \textbf{No.} & \textbf{Data 1} & \textbf{Data 2} \\ \hline
%     1 & a1 & b1 \\ \hline
%     2 & a2 & b2 \\ \hline
%    \end{tabular}
%\end{center}
%\caption{Table 1.}
%\label{Tab1}
%\end{table}
%
%\noindent To refer to the table
%
%\begin{verbatim}
%\textbf{Table. \ref{Tab1}}
%\end{verbatim}
%
%\noindent This will produce \textbf{Table. \ref{Tab1}}.

\cleardoublepage