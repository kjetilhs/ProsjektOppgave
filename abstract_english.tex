\pagestyle{empty}
\begin{abstract}
Automatic landing of a fixed wing \acrfull{uav} in a net on a ship require an accurate positioning system. There exist today high-end systems with such capability for special applications, e.g military systems and costly commercial systems. To increase general availability the system must consist of low-cost components. An alternative is the use of low-cost \gls{gnss} receivers and apply \gls{rtk-gps}, which can provide centimeter level position accuracy. However the processing time for the \gls{rtk-gps} system results in degraded accuracy when exposed to highly dynamical behaviour.

This work present two alternative software and hardware position systems suitable for use in navigation system which apply \gls{rtk-gps}, namely \gls{rtklib} with two Ublox Lea M8T receivers and two Piksi systems. Both the Piksi and the Ublox receivers are single-frequency \gls{gnss} receivers. The Piksi supports \gls{rtk-gps}, while the Ublox can send raw \gls{gnss} data to \gls{rtklib}. The \gls{rtklib} version used in this work is an altered version from the latest release. These systems will in this work be compared and their individual capability to provide accurate position estimate will be evaluated. 

The \gls{rtk-gps} system is implemented in DUNE (DUNE:Unified Navigation Environment) framework running on an embedded payload computer on-board the \gls{uav}.

The performance of these position systems are in this work investigated by experimental testing. These tests of the navigation system was successfully performed, and have proven that the navigation system is sufficient from integration into a control and guidance system.

Of the two positioning system the \gls{rtklib} combined with the Ublox LEA M8T receiver proved superior to the Piksi system.
\end{abstract}
%\keywords{UAV,RTK-GPS,}